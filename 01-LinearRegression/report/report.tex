\documentclass[a4paper, 11pt]{article}
\usepackage{graphicx}
\usepackage{amsmath}
\usepackage[pdftex]{hyperref}

% used to list matlab code - check mcode.zip mcode.sty
\usepackage[framed,numbered,autolinebreaks,useliterate]{mcode}


% Lengths and indenting
\setlength{\textwidth}{16.5cm}
\setlength{\marginparwidth}{1.5cm}
\setlength{\parindent}{0cm}
\setlength{\parskip}{0.15cm}
\setlength{\textheight}{22cm}
\setlength{\oddsidemargin}{0cm}
\setlength{\evensidemargin}{\oddsidemargin}
\setlength{\topmargin}{0cm}
\setlength{\headheight}{0cm}
\setlength{\headsep}{0cm}

\renewcommand{\familydefault}{\sfdefault}

\title{Machine Learning 2013: Project 1 - Regression Report}
\author{fregli@student.ethz.ch\\ ganzm@student.ethz.ch\\ sandrofe@student.ethz.ch\\}
\date{\today}

\begin{document}
\maketitle

\section*{Experimental Protocol}
\label{sec:exp-protocl}
%Suppose that someone wants to reproduce your results. Briefly describe the steps used to obtain the predictions starting from the raw data set downloaded from the project website. Use the following sections to explain your methodology. Feel free to add graphs or screenshots if you think it's necessary. The report should contain a maximum of 2 pages.

This project was performed using Matlab only. To reproduce test results presented in this report the following steps have to be taken:

\begin{itemize}
\item Unzip the sourcefolder containing Matlab code and data sets
\item Run learn.m to optain both files testresult.csv and validationresult.csv
\end{itemize}

\section{Tools}
%Which tools and libraries have you used (e.g. Matlab, Python with scikit-learn, Java with Weka, SPSS, language x with library y, $\ldots$). If you have source-code (Matlab scripts, Python scripts, Java source folder, \dots), make sure to submit it on the project website together with this report. If you only used command-line or GUI-tools describe what you did.

As stated in Section \nameref{sec:exp-protocl} the only tool which is needed is Matlab - no fancy special commands, no additional libraries are required.

\section{Algorithm}
%Describe the algorithm you used for regression (e.g. ordinary least squares, ridge regression, $\ldots$)

The algorithm performs the following steps an can be started by running learn.m

\begin{itemize}
\item Read csv file
\item Normalize training input and result data
\item create feature vectors
\item perform cross validation and obtain weight vector $w$
\item generate validation and testing output
\end{itemize}

\subsection{Which part of the code does what}

\paragraph{learn.m}
This is the main script which can be run to obtain the submitted results.
\paragraph{trainData.m}
Performs Ridge Regression and solves the following problem

$ \min \limits_w \sum \limits_{i=1}^n \left(y_i - w^Tx_i\right)^2 + \lambda ||w||_2^2$

using the exact solution:


$ w = \left(X^T X + \lambda I   \right)^{-1} X^T y$

As a result we obtain the weight vector $w$ and an the RMSE.

\paragraph{extractFeatures.m}

This is the part of the code where features are created and transformed.


\section{Features}
% Did you construct any new features? What feature transforms did you use?

\textbf{TODO}

\subsection{Find good features}
\label{subsec:findgoodfeatures}

\textbf{TODO}

\begin{figure}
\begin{center}
\begin{lstlisting} 
x = trainingData(:,1:14);
y = transform(trainingData(:,15));

% prepare features
xfeatures = [y, x, x.^2, x.^3, x.^4, sqrt(x), log2(x)];

% calculate correlation
corr = corr(xfeatures, xfeatures);

% only consider how good the y feature correlates
c = corr(:,1);
\end{lstlisting}
\end{center}
\caption{Code Sample to find correlation}
\label{lst:matlab-correlation}
\end{figure}



\section{Parameters}
%How did you find the parameters of your model? (What parameters have you searched over, cross validation procedure, $\ldots$)

\subsection{Parameter $\lambda$}

By iteratively applying cross validation best parameter $\lambda$ is determined as seen in \nameref{lst:matlab-crossvalidation}. We chose an arbitrary number of parameters (50 in this sample) and apply cross validation. The parameter which yields the smallest error is then chosen.


\begin{figure}
\begin{center}
\begin{lstlisting} 
% perform crossvalidation
lambdaValues = logspace(-6, 2, 50); % hyper parameter
meanErrs = zeros(size(lambdaValues));

for i=1:size(lambdaValues,2)
    [meanErrs(i), W, errorTest] = crossvalidation(Xnorm, Ynorm, lambdaValues(i));
end
\end{lstlisting}
\end{center}
\caption{Code Sample to find best $\lambda$}
\label{lst:matlab-crossvalidation}
\end{figure}

\section{Lessons Learned} 
Most of the time spent for the project was used to search for good features. At the start of the project  it was mostly unclear how to find such features. Getting a test score which meets our expectation by simply try and error features did not yield the expected result.

We later automated our search for good features by applying the methods described in \nameref{subsec:findgoodfeatures}. This lead to much better result.

%What other algorithms, tools or methods did you try out that didn't work well?
%Why do you think they performed worse than what you used for your final submission?

\end{document} 
